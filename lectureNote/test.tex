\documentclass{article}
\usepackage{graphicx}
\graphicspath{{images/}}
\usepackage{amssymb}

\title{Proves and Problem Solving}

\begin{document}
\maketitle


\section{Upper bounds and Least upper bound}



\subsection{Bounded Sequence}

We say that a sequence $(a_{n})$ is \textbf{bounded} if the set of values {a1,a2,...} is a bounded set. 
\\i.e. thereare m,M such that $m \leq a_{n} \leq M$ for all n.
\textbf{Proposition 6.1}\\
Suppose the sequence $(a_{n})$ converges. Then it is bounded.\\
\\
\textbf{Proposition 6.3}\\
Suppose that $x_{n} \rightarrow L$ as $n \rightarrow \infty$ and that $k \in \mathbb{N}$. Then $x{_{n}^{k}} \rightarrow L^{k}$ as $n \rightarrow \infty$.\\




\subsection {Application: the existence of roots}
\textbf{Theorem 6.6}\\
Let $x>0$ and $k \in \mathbb{N}$. Then there is a unique $y>0$ such that $y^{k}=x$.\\

\subsection {Infinite limits}
\textbf{Definition 6.4}\\
Let ${x_{n}}$ be a sequence of real numbers.\\
(a) We say $(x_{n})$ tends to $\infty$ or \textbf{diverges} to $\infty$, and write $x_{n}\rightarrow \infty$ as $n \rightarrow \infty$ if for all $M > 0$, there exists $N \in \mathbb{N} such that $n>N$ implies $x_{n}\geq M$.\\

(b) Similar for $x_{n} \rightarrow -\infty$





\end{document}