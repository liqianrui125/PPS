\documentclass[12pt]{article}

\usepackage{amsmath}
%\widebar command from https://tex.stackexchange.com/a/60253/286191
\makeatletter
\let\save@mathaccent\mathaccent
\newcommand*\if@single[3]{%
  \setbox0\hbox{${\mathaccent"0362{#1}}^H$}%
  \setbox2\hbox{${\mathaccent"0362{\kern0pt#1}}^H$}%
  \ifdim\ht0=\ht2 #3\else #2\fi
  }
%The bar will be moved to the right by a half of \macc@kerna, which is computed by amsmath:
\newcommand*\rel@kern[1]{\kern#1\dimexpr\macc@kerna}
%If there's a superscript following the bar, then no negative kern may follow the bar;
%an additional {} makes sure that the superscript is high enough in this case:
\newcommand*\widebar[1]{\@ifnextchar^{{\wide@bar{#1}{0}}}{\wide@bar{#1}{1}}}
%Use a separate algorithm for single symbols:
\newcommand*\wide@bar[2]{\if@single{#1}{\wide@bar@{#1}{#2}{1}}{\wide@bar@{#1}{#2}{2}}}
\newcommand*\wide@bar@[3]{%
  \begingroup
  \def\mathaccent##1##2{%
%Enable nesting of accents:
    \let\mathaccent\save@mathaccent
%If there's more than a single symbol, use the first character instead (see below):
    \if#32 \let\macc@nucleus\first@char \fi
%Determine the italic correction:
    \setbox\z@\hbox{$\macc@style{\macc@nucleus}_{}$}%
    \setbox\tw@\hbox{$\macc@style{\macc@nucleus}{}_{}$}%
    \dimen@\wd\tw@
    \advance\dimen@-\wd\z@
%Now \dimen@ is the italic correction of the symbol.
    \divide\dimen@ 3
    \@tempdima\wd\tw@
    \advance\@tempdima-\scriptspace
%Now \@tempdima is the width of the symbol.
    \divide\@tempdima 10
    \advance\dimen@-\@tempdima
%Now \dimen@ = (italic correction / 3) - (Breite / 10)
    \ifdim\dimen@>\z@ \dimen@0pt\fi
%The bar will be shortened in the case \dimen@<0 !
    \rel@kern{0.6}\kern-\dimen@
    \if#31
      \overline{\rel@kern{-0.6}\kern\dimen@\macc@nucleus\rel@kern{0.4}\kern\dimen@}%
      \advance\dimen@0.4\dimexpr\macc@kerna
%Place the combined final kern (-\dimen@) if it is >0 or if a superscript follows:
      \let\final@kern#2%
      \ifdim\dimen@<\z@ \let\final@kern1\fi
      \if\final@kern1 \kern-\dimen@\fi
    \else
      \overline{\rel@kern{-0.6}\kern\dimen@#1}%
    \fi
  }%
  \macc@depth\@ne
  \let\math@bgroup\@empty \let\math@egroup\macc@set@skewchar
  \mathsurround\z@ \frozen@everymath{\mathgroup\macc@group\relax}%
  \macc@set@skewchar\relax
  \let\mathaccentV\macc@nested@a
%The following initialises \macc@kerna and calls \mathaccent:
  \if#31
    \macc@nested@a\relax111{#1}%
  \else
%If the argument consists of more than one symbol, and if the first token is
%a letter, use that letter for the computations:
    \def\gobble@till@marker##1\endmarker{}%
    \futurelet\first@char\gobble@till@marker#1\endmarker
    \ifcat\noexpand\first@char A\else
      \def\first@char{}%
    \fi
    \macc@nested@a\relax111{\first@char}%
  \fi
  \endgroup
}
\makeatother

\renewcommand{\bar}{\widebar}

\usepackage[tmargin=2cm,rmargin=2.5cm,lmargin=2.5cm,bmargin=2cm,footskip=0.4cm]{geometry} 
% Top margin, right margin, left margin, bottom margin, footnote skip
\usepackage[utf8]{inputenc}
\usepackage{biblatex}
\addbibresource{./reference/reference.bib}
% linktocpage shall be added to snippets.
\usepackage{hyperref,theoremref}
\hypersetup{
	colorlinks, 
	linkcolor={red!40!black}, 
	citecolor={blue!50!black},
	urlcolor={blue!80!black},
	linktocpage % Link table of content to the page instead of the title
}

\usepackage{blindtext}
\usepackage{titlesec}
\usepackage{amsthm}
\usepackage{thmtools}
\usepackage{amsmath}
\usepackage{amssymb}
\usepackage{graphicx}
\usepackage{titlesec}
\usepackage{xcolor}
\usepackage{multicol}
\usepackage{hyperref}
\usepackage{import}
\usepackage{bm}
\usepackage{breqn}


\newtheorem{theorem}{Theorema}[section]
\newtheorem{lemma}[theorem]{Lemma}
\newtheorem{corollary}{Corollarium}[section]
\newtheorem{proposition}{Propositio}[theorem]
\theoremstyle{definition}
\newtheorem{definition}{Definitio}[section]

\theoremstyle{definition}
\newtheorem{axiom}{Axioma}[section]

\theoremstyle{remark}
\newtheorem{remark}{Observatio}[section]
\newtheorem{hypothesis}{Coniectura}[section]
\newtheorem{example}{Exampli Gratia}[section]

% Proof Environments
\newcommand{\thm}[2]{\begin{theorem}[#1]{}#2\end{theorem}}

%TODO mayby proof environment shall have more margin
\renewenvironment{proof}{\vspace{0.4cm}\noindent\small{\emph{Demonstratio.}}}{\qed\vspace{0.4cm}}
% \renewenvironment{proof}{{\bfseries\emph{Demonstratio.}}}{\qed}
\renewcommand\qedsymbol{Q.E.D.}
% \renewcommand{\chaptername}{Caput}
\renewcommand{\contentsname}{Index Capitum} % Index Capitum 
\renewcommand{\emph}[1]{\textbf{\textit{#1}}}
\renewcommand{\ker}[1]{\operatorname{Ker}{#1}}

%\DeclareMathOperator{\ker}{Ker}

% New Commands
\newcommand{\bb}[1]{\mathbb{#1}} %TODO add this line to nvim snippets

% ALGEBRA
\newcommand{\orb}[2]{\text{Orb}_{#1}({#2})}
\newcommand{\stab}[2]{\text{Stab}_{#1}({#2})}
\newcommand{\im}[1]{\text{im}{\ #1}}
\newcommand{\se}[2]{\text{send}_{#1}({#2})}

%STATISTICS
\newcommand{\var}[1]{\text{Var}(#1)}
\newcommand{\ud}[1]{\underline{#1}}
\newcommand{\cor}[1]{\text{Cor}(#1)}
\newcommand{\std}[1]{\text{Std}(#1)}
\newcommand{\ste}[1]{\text{S.E.}(#1)}


\title{PPS Cheatsheet}
\author{Qianrui Li}
\date{\today}
\begin{document}
\maketitle


\section{Logic}

\begin{definition}[Disjunction]
	$A\lor B$ is true if either A or B is true.
\end{definition}

\begin{definition}[Converse]
	Converse to a conditional $A \implies B$ is $B \implies A$.
\end{definition}

\begin{definition}[Contrapositive]
	Contrapositive to a conditional $A \implies B$ is $\bar{B} \implies \bar{A}$.
\end{definition}

\section{Real Numbers}

\begin{definition}[Field]
	A field is a set $F$ with two operations, addition and multiplication, such that
	\paragraph{Addition}
	\begin{enumerate}
		\item $a+b \in F$ for all $a,b \in F$.
		\item $a+b = b+a$ for all $a,b \in F$.
		\item $(a+b)+c = a+(b+c)$ for all $a,b,c \in F$.
		\item There exists an element $0 \in F$ such that $a+0 = a$ for all $a \in F$.
		\item For each $a \in F$, there exists an element $-a \in F$ such that $a+(-a) = 0$.
	\end{enumerate}
	\paragraph{Multiplication}
	\begin{enumerate}
		\item $a \cdot b \in F$ for all $a,b \in F$.
		\item $a \cdot b = b \cdot a$ for all $a,b \in F$.
		\item $(a \cdot b) \cdot c = a \cdot (b \cdot c)$ for all $a,b,c \in F$.
		\item There exists an element $1 \in F$ such that $a \cdot 1 = a$ for all $a \in F$.
		\item For each $a \in F$, if $a \neq 0$, there exists an element $a^{-1} \in F$ such that $a \cdot a^{-1} = 1$.
	\end{enumerate}
	\paragraph{Distributive Law}
	\begin{enumerate}
		\item $a \cdot (b+c) = a \cdot b + a \cdot c$ for all $a,b,c \in F$.
	\end{enumerate}
\end{definition}

\section{Inequality}

\begin{definition}[Ordered Field]
	An ordered field is a field $F$ with a relation $<$ such that
	\begin{enumerate}
		\item Exactly one of is true: $a < 0$, $a = 0$, or $0 < a$ 
		\item $b<a $ imples $ -a<-b$.
		\item If $a < b$, then $a+c < b+c$ for all $a,b,c \in F$.
		\item $a>b$ and $b>0$ implies that $ab>0$.
		\item If $a < b$ and $b < c$, then $a < c$ for all $a,b,c \in F$.
	\end{enumerate}
\end{definition}
\begin{theorem}[AM-GM Inequality]
	For any non-negative real numbers $a_1, a_2, \dots, a_n$, we have
	\begin{equation}
		\frac{a_1 + a_2 + \dots + a_n}{n} \geq \sqrt[n]{a_1 a_2 \dots a_n}
	\end{equation}
\end{theorem}
%cauchy-schwarz Inequality
\begin{theorem}[Cauchy-Schwarz Inequality]
	For any real numbers $\bm{u} = [x_1, x_2, \dots, x_n]$ and $\bm{v} = [y_1, y_2, \dots, y_n]$, we have
	\begin{equation}
		x_1y_1+x_2y_2+\cdots+x_ny_n \leq \sqrt{x_1^2+x_2^2+\cdots+x_n^2}\sqrt{y_1^2+y_2^2+\cdots+y_n^2} \implies \bm{u}\cdot \bm{v} \leq ||\bm{u}|| ||\bm{v}||
	\end{equation}
\end{theorem}
\begin{theorem}[Triangular Inequality]
	$$|x+y|\leq  |x|+|y|;\hspace{1cm} ||x|-|y|| \geq |x+y|$$
\end{theorem}
\begin{theorem}[Formula for Geometric Series]
	For any real number $x$ and integer $n \geq 0$, we have
	\begin{equation}
		a + ax + ax^2 + \dots + ax^n = \frac{a-x^{n+1}}{1-x}
	\end{equation}
\end{theorem}

\section{Completeness Axiom}

\begin{axiom}[Completeness of Real number]
	Every nonempty set of real numbers that is bounded above has a least upper bound.
\end{axiom}

\begin{definition}[Monotone Sequence]
	A sequence $(a_n)^{\infty}_{n=1}$ is increasing if $a_n \leq a_{n+1}$ for all $n \geq 1$. A sequence $(a_n)^{\infty}_{n=1}$ is decreasing if $a_n \geq a_{n+1}$ for all $n \geq 1$.
	A seqence is monotone if it is increasing or decreasing.
\end{definition}
\begin{theorem}[Monotone Convergence Theorem]
	A bounded above increasing sequence of real number converges; likewise a bounded below decreasing real sequence converges.
\end{theorem}

\begin{definition}[Convergence of series (Partial Sum)]
	Given a sequence $(a_j)$, the infinite series $\sum_{j=1}^{\infty} a_j$ converges to $s$ if the sequence of partial sums 
	$$s_n = \sum ^{n}_{j=1}a_j$$
	converges as $n \to \infty$. If $(s_n)$ converges we denote its limit by $\sum^{\infty}_{j=1}a_j$
\end{definition}
\begin{definition}[Base of Natural Logarithm]	
	\begin{equation}
		e=\sum^{\infty}_{n=0}\frac{1}{n!}=\lim_{n\to\infty}\left(1+\frac{1}{n}\right)^n
	\end{equation}
\end{definition}

\section{Polynomials}

\begin{definition}[N-degree Complex Polynomial]
	For $n\in \bb{N}$, an \emph{n-degree complex polynomial} is a function of the form 
	\begin{equation}
		p(z)=a_nz^n+a_{n-1}z^{n-1}+\cdots+a_1z+a_0
	\end{equation}
	where $a_n\neq 0$ and $a_i \in \bb{C}$. A root of $p$ is number $\alpha$ such that $p(\alpha)=0$. 
\end{definition}

\begin{theorem}[Root Coefficient Theorem]
	Let $p(z) = x^n+a_{n-1}x^{n-1} + \cdots + a_1x+a_0$ have roots $r_1, r_2, \cdots r_n$, then
	\begin{equation}
		r_1+r_2+\cdots+r_n=-a_{n-1};\hspace{1cm} r_1r_2\cdots r_n=(-1)^na_0
	\end{equation}
	In general, let $s_j$ denote the sum of all products of j-tuples of the roots (e.g., $s_2 = r_1r_2+r_1r_3+r_2r_3 \cdots$), then 
	\begin{equation}
		s_j = (-1)^ja_{n-j}
	\end{equation}
\end{theorem}
\begin{theorem}[Fundamental Theorem of Arithmetic]
	Let $n\leq 2$ be an integer.

	\textbf{Existance} $n$ is equal to the product of prime number $n=p_1^{r_1}p_2^{r_2}\cdots p_k^{r_k} $, where $p_1<p_2<\cdots<p_k$ and $r_i>0$ for all $i$.

	\textbf{Uniqueness} The factorisation is unique, i.e., if $q_1^{s_1}q_2^{s_2}\cdots q_l^{s_l}=n=p_1^{r_1}p_2^{r_2}\cdots p_k^{r_k} $ are ``two`` prime factorisations, then $k=l$ and $p_i=q_i$, $r_i=s_i$ for all $i$.
\end{theorem}

\section{Number Theory}

\begin{theorem}
	Let $m\geq 2$ be a natural number, $\bb{Z}/m$ is a field if and only if $m$ is a prime.
\end{theorem}
\begin{theorem}[Fermat's Little Theorem]
	Let $p$ be a prime number, then for any integer $a$, if $p$ does not divide a, we have
	\begin{equation}
		a^{p-1}\equiv 1 \mod p
	\end{equation}
\end{theorem}

\begin{theorem}
Let $n\in \bb{N}$ and $p$ be a prime. If $n$ and $p-1$ are coprime and $p$ divides not $b$, the equation 
\begin{equation}
	x^n\equiv b \mod p
\end{equation}
has exactly one solution $x\in \{0, 1, \cdots, p-1\}$.
\end{theorem}

\section{Relation and Function}

\begin{definition}[Cartesian Product]
	Let $X$ and $y$ be sets; their Cawrtesian Product is the set of ordered pairs 
	\[ 
		X\times Y = \{(x,y): x\in X, y\in Y\}
	\]
\end{definition}
\begin{definition}[Injectivity and Surjectivity]
	Let $f:X \rightarrow y$ be a function. 
	\begin{enumerate}
		\item		It is \emph{injective} if and only if $f(a)=f(b) \implies a=b$
		\item It is \emph{surjective} if and only if for all $y\in Y$ we have $x\in X$ such that $f(x)=y$	
		\item	It is \emph{bijective} if and only if it is both injective and surjective.
	\end{enumerate}
\end{definition}

\begin{definition}[Images and Preimages]
	Let $f:X \rightarrow Y$ be a function. For $A \subseteq X$, the \emph{image} of $A$
	\[
		f(A) = \{f(x): x\in A\}\subseteq Y
	\]
	For $B \subseteq Y$, the \emph{preimage} of $B$ is 
	\[
		f^{-1}(B) = \{x\in X: f(x)\in B\}\subseteq X
	\]
\end{definition}

\begin{definition}[Cardinality]
	Tow sets $A, B$ have the same cardinality, (i.e., isomorphic) if there is a bijection $f:A \rightarrow B$. We write $|A|=|B|$ or $A \cong B$. If there is an injective map, we write $|A| \leq |B|$, and if $|A| \leq |B|$ and there is no injective map from $B$ to $A$, we write $|A| < |B|$.

	When $|A| \leq \bb{N}$, we say $A$ is countable.
\end{definition}

\begin{definition}
	Let $S=\{a_1, a_2, \cdots, a_n\}$ be a set of $n$ distinct objects. An ordering (or arrangement) of $S$ is a sequence $(a_1, \cdots, a_n$ in which each element of $S$ appears exactly once.
\end{definition}

\begin{theorem}
For $n\geq 0$, 
\[
	2^n = \sum_{k=0}^n \binom{n}{k}
\]
\end{theorem}

\begin{definition}
	Given $n\in \bb{N}, k \in \bb{N}$, with $k \geq 2$ and non-negative integers $r_1, r_2, \cdots r_k$ such that $r_1+\cdots r_k=n$, we denote the number of ordered partitions $(A_1, \cdots, A_k)$ of set $S$ such that $|A_i|=r_i$ by 
	\[ 
		\binom{n}{r_1, r_2, \cdots, r_k} = \frac{n!}{r_1!r_2!\cdots r_k!}
	\] 
\end{definition}

\section{Permutaion}

\begin{definition}
	Given $n \in \bb{N}$, denote by $S_n$ the set of all bijections $\{1,2,3,4, \cdots,n\} \rightarrow \{1,2,3, \cdots, 4\}$. We call $S_n$ permutation of the set $\{1,2,3,\cdots,n\}$. (Which is also called the symmetric group of degree $n$)
\end{definition}
\end{document}
